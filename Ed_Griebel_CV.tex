% !TEX TS-program = xelatex
% !TEX encoding = UTF-8 Unicode
% -*- coding: UTF-8; -*-
% vim: set fenc=utf-8

%%%%%%%%%%%%%%%%%%%%%%%%%%%%%%%%%%%%%%%%%%%%%%%%%%%%%%%%%%%%%%%%%
%% SIMPLE-RESUME-CV
%% <https://github.com/zachscrivena/simple-resume-cv>
%% This is free and unencumbered software released into the
%% public domain; see <http://unlicense.org> for details.
%%%%%%%%%%%%%%%%%%%%%%%%%%%%%%%%%%%%%%%%%%%%%%%%%%%%%%%%%%%%%%%%%

% See "README.md" for instructions on compiling this document.

\documentclass[letterpaper,MMMMyyyy,nonstopmode]{simpleresumecv}
% Class options:
% a4paper, letterpaper, nonstopmode, draftmode
% MMMyyyy, ddMMMyyyy, MMMMyyyy, ddMMMMyyyy, yyyyMMdd, yyyyMM, yyyy

%%%%%%%%%%%%%%%%%%%%%%%%%%%%%%%%%%%%%%%%%%%%%%%%%%%%%%%%%%%%%%%%%
%% PREAMBLE.
%%%%%%%%%%%%%%%%%%%%%%%%%%%%%%%%%%%%%%%%%%%%%%%%%%%%%%%%%%%%%%%%%

% CV Info (to be customized).
\newcommand{\CVAuthor}{Edward E Griebel Jr.}
\newcommand{\CVTitle}{Ed Griebel CV}
\newcommand{\CVNote}{CV written in \LaTeX, compiled on {\today}}
\newcommand{\CVWebpage}{https://www.edgriebel.com/}

% PDF settings and properties.
\hypersetup{
pdftitle={\CVTitle},
pdfauthor={\CVAuthor},
pdfsubject={\CVWebpage},
pdfcreator={XeLaTeX},
pdfproducer={},
pdfkeywords={},
unicode=true,
bookmarks=true,
bookmarksopen=true,
pdfstartview=FitH,
pdfpagelayout=OneColumn,
pdfpagemode=UseOutlines,
hidelinks,
breaklinks}

% Shorthand.
\newcommand{\Code}[1]{\mbox{\textbf{#1}}}
\newcommand{\CodeCommand}[1]{\mbox{\textbf{\textbackslash{#1}}}}
\newcommand{\comment}[1]{\ignorespaces} %% https://tex.stackexchange.com/a/276698/143811

%%%%%%%%%%%%%%%%%%%%%%%%%%%%%%%%%%%%%%%%%%%%%%%%%%%%%%%%%%%%%%%%%
%% ACTUAL DOCUMENT.
%%%%%%%%%%%%%%%%%%%%%%%%%%%%%%%%%%%%%%%%%%%%%%%%%%%%%%%%%%%%%%%%%

\begin{document}

%% Use Sans font
%% Default Sans font looks not too sharp, Company name in Serif font
%%\renewcommand\familydefault{\sfdefault}

%% \UseSecondaryFont %% <-- Doesn't work, doesn't switch font

%%%%%%%%%%%%%%%
% TITLE BLOCK %
%%%%%%%%%%%%%%%

\Title{\CVAuthor}

\begin{SubTitle}
%%
%% REPLACE WITH ADDRESS
%%
%% \textit{<my address>}
\par 
\href{mailto:edgriebel@gmail.com}
{edgriebel@gmail.com}
%%
%% REPLACE WITH CELLPHONE 
%%
\,\SubBulletSymbol\, 585-705-1093
\,\SubBulletSymbol\,
\href{\CVWebpage}
{\url{\CVWebpage}}

\href{https://github.com/edgriebel}{https://github.com/edgriebel}
\,\SubBulletSymbol\,
\href{https://www.linkedin.com/in/edgriebel}{https://linkedin.com/in/edgriebel}
\end{SubTitle}

\begin{Body}
\Section{Summary}{Summary}{PDF:Summary}

Language-Agnostic Full Stack Developer with over 20 years of success creating software solutions.
Wide breadth of experience in multiple industries, languages, tools, and third-party applications demonstrates success.
Excitement about creating code to deliver high quality systems that support business growth.
\iffalse
\Entry
Full Stack Java architect, developer, and consultant, specializing in Java/Oracle/JavaScript
solutions and integrations.
\Gap
\textit{Technical Specialties}: Java, J2EE/JavaEE, SQL, PL/SQL, JavaScript, CSS, JMS, jQuery, jQuery UI, RabbitMQ, Oracle, Cognos, JBoss, WebLogic, AppDynamics.
\Gap
\textit{Business Specialties}: Agile team leadership, customer-focused development, requirements gathering and analysis, technical liaison, systems integration, business intelligence
\fi
%%%%%%%%%%%%%%%%%%%%%%%%%%%%%%%%%%
%% PROFESSIONAL WORK EXPERIENCE %%
%%%%%%%%%%%%%%%%%%%%%%%%%%%%%%%%%%

\Section
{Professional\newline
Work Experience}
{Professional Work Experience}
{PDF:ProfessionalWorkExperience}
\Entry
\href{http://www.gocatalant.com}
{\textbf{Catalant Technologies}}
\hfill
\DatestampYM{2019}{06} -- Present

Principal Software Engineer
\hfill
Rochester, New York

\begin{Detail}
\iffalse
Catalant helps companies accelerate growth, drive efficiency, and innovate rapidly with a SaaS platform that organizes work and dynamically matches people to opportunities. 
\fi

\BulletItem
Determine performace issues impacting system scalability and work with technical owners for prioritization and resolution
\BulletItem
Create and build consensus on engineering best practices to improve developer productivity and long-term supportability
\BulletItem
Create new features for customers and tools to enhance developer productivity
\BulletItem
Mentor and guide junior developers in Catalant's development environment and general software engineering practices


\Gap
Technologies: Python, Flask, JavaScript, MySQL, Bootstrap, Docker

\end{Detail}

\BigGap

\Entry
\href{http://ist.rit.edu/}
{\textbf{Rochester Institute of Technology}}
\hfill
\DatestampYM{2017}{08} --
Present

%%\Gap
%%\BulletItem
Adjunct Professor
\hfill 
Rochester, New York

\begin{Detail}
\BulletItem
Lecturer for undergraduate couses in Java, DBMS/SQL, and Software Engineering in Industry
\end{Detail}

\BigGap

\Entry
\href{http://www.paychex.com}
{\textbf{Paychex}}
\hfill
\DatestampYM{2013}{06} -- \DatestampYM{2019}{06}

Middleware Architect %% Crappy Name!!
\hfill
Rochester, New York

\begin{Detail}
\iffalse
At Paychex I work with cross-functional teams to build: an inventory system to track hosts, managed VMs, databases, and middleware appliances; machine learning systems to efficiently repack databases and to predict system utilization based on user metrics; a ground-up implementation of the Netflix Chaos Monkey; implemented AppDynamics on over 20,000 JVMs; and a system to collect and monitor JVM metrics and send alerts via SNMP and SMTP.
\fi

\BulletItem
Created system to rebalance 20,000 databases among 200 pools in Azure cloud using algorithmic, statistical, and machine learning techniques with TensorFlow, Pandas, NumPy, Jupyter, and scikit-learn in a Python pipeline.

\BulletItem
Built IT inventory system that provides a single source of truth for virtualized and physical hosts, JVMs, CLRs, databases, and middleware infrastructure with a HTML5 webapp and REST front-end. Relationships were sourced from external systems including AppDynamics, Puppet/Facter, ServiceNow, and Oracle OEM with a custom export to ServiceNow’s CMDB. 

\BulletItem
Custom Chaos Monkey randomly kills LDAP and SiteMinder middleware processes, and Java microservices daily in production and non-production environments. The Chaos Monkey is monitored and controlled by a Spring Boot web application and Splunk dashboard.

\BulletItem
Implemented AppDynamics on over 20,000 JVMs to replace our custom built system that collected and monitored JVM metrics and send alerts via SNMP and SMTP.

\Gap
Technologies: Java, JavaScript, JQuery-UI, Oracle PL/SQL, Scikit-Learn, TensorFlow, Python, middleware, AppDynamics, Application Performance Monitoring, ClearCase, Ant, Git, JUnit, Vagrant
\end{Detail}

\BigGap

\iffalse %% ORIGINAL PAYCHEX SECTION
\Entry
\href{http://www.paychex.com/}
{\textbf{Paychex}}
\hfill 
Rochester, New York

%%\Gap
%%\BulletItem
Middleware Architect
\hfill
\DatestampYM{2013}{06} --
Present

\begin{Detail}
\BulletItem
Created a process to rebalance 20,000 databases between 200 compute servers in the cloud. The goal is to maximize density by reducing cost while minimizing overall spikes in usage that affect customers' experience. Using algorithmic, statistical, and machine learning techniques including TensorFlow, Pandas, NumPy, and scikit-learn in Python.
\BulletItem
Created an internal IT Inventory system in HTML5/JavaScript/Java/RabbitMQ to provide single source of truth and display relationships between virtual and physical hosts, JVM/CLR runtimes, databases, and middleware infrastructure.
\BulletItem
Lead developer on system to gather and present metrics from hosts, JVMs, and middleware components, used to send alerts for problems and provide a single location to view runtime metrics.
\BulletItem
Created tools
in Python to automate configuration and installation of AppDynamics on thousands of Weblogic servers using Weblogic WLST
scripting tool.
\BulletItem
Member of cross-functional team for selection, prototyping, and implementation of AppDynamics application monitoring tool. 

\Gap
Technologies: Java, JavaScript, JQuery-UI, Oracle PL/SQL, Scikit-Learn, TensorFlow, Python, middleware, AppDynamics, Application Performance Monitoring, ClearCase, Ant, Git, JUnit, Vagrant
\end{Detail}
\fi

\BigGap

\Entry
\textbf{Axeda} (division of PTC)
\hfill 
\DatestampYM{2010}{06} --  \DatestampYM{2013}{06}

%%\Gap
%%\BulletItem
Senior Software Engineer, R\&D
\hfill
Rochester, New York

\begin{Detail}
\BulletItem
Full stack Java/Oracle/ActiveMQ JMS developer for IoT system.
\BulletItem
Lead for Cognos Business Intelligence integration with IoT data. Integrated Groovy with Cognos to script report creation, execution, publishing, and administration.
\BulletItem
Internal consulting for new and existing solutions for professional services and product management teams.

\Gap
Technologies: Java, JavaScript, JBoss, Oracle PL/SQL, Hudson/Jenkins, JUnit, Cognos, Business Intelligence, Maven, SVN
\end{Detail}

\BigGap
\Entry
\href{http://www.paychex.com/}
{\textbf{Paychex, Inc}}
\hfill 
%%\DatestampYM{2005}{03} --  \DatestampYM{2010}{06}
\DatestampY{2005} -- \DatestampY{2010}

%%\Gap
%%\BulletItem
Lead Developer
\hfill
Rochester, New York

\begin{Detail}
\BulletItem
Lead team converting two existing external Java/JSP/Oracle applications to Web 2.0 and creating new external application from existing internal-only functions. Leveraged Hudson for Continuous Integration and automated turnover to QA.
\BulletItem
Mentored new developers, delivered technical training, and provided answers to challenging technical questions for development team.
\BulletItem
Member of architecture team to research and set five-year strategy for web application strategy.

\Gap
Technologies: Java, JavaScript, JSP, Struts, WebLogic, Oracle PL/SQL, Hudson/Jenkins, JUnit, ClearCase
\end{Detail}

\BigGap
\Entry
\href{http://www.frontier.com/}
{\textbf{Frontier Communications}}
\hfill 
%% \DatestampYM{2001}{11} --  \DatestampYM{2005}{03}
\DatestampY{2001} -- \DatestampY{2005}

%%\Gap
%%\BulletItem
Senior Software Developer
\hfill
Rochester, New York

\begin{Detail}
\Item
Created Java applications to build DSL circuits and track commissions\comment{; CruiseControl used for Continuous Integration}. 
Automated generation of \comment{20} Excel spreadsheets from Hyperion EssBase saving two FTEs weekly.

%% \Gap
%% Technologies: Java, JSP, Struts, Oracle PL/SQL, Continuous Integration, JUnit, Excel VBA, EssBase Data Warehouse, Ant
\end{Detail}

\BigGap
\Entry
\href{http://www.globalcrossing.com/}
{\textbf{Global  Crossing} (division of Level3)}
\hfill 
%% \DatestampYM{1999}{03} --  \DatestampYM{2001}{11}
\DatestampY{1999} -- \DatestampY{2001}

%%\Gap
%%\BulletItem
Senior/Lead Developer
\hfill
Rochester, New York

\begin{Detail}
\Item
Lead multi-site teams to create Java and Perl CGI telecom applications.
%%Lead for multi-site team creating Java application to reconcile network and order entry inventory and Perl CGI application to automate creating circuits on ATM and Frame Relay networks.

%% \Gap
%% Technologies: Perl, CGI, Java, Sybase, Solaris, Ant
\end{Detail}

\BigGap
\Entry
\href{http://www.jpmorganchase.com/}
{\textbf{JP Morgan Chase}}
\hfill 
%% \DatestampYM{1995}{06} -- \DatestampYM{1999}{02}
\DatestampY{1995} - \DatestampY{1999}

%%\Gap
%%\BulletItem
Lead Developer
\hfill
New York, New York

\begin{Detail}
\Item
Created trading systems in C/C++ and tactical applications for equity derivatives and HFT trading desks.

%% Tech from 20 years ago would probably be considered not relevant
%% \Gap
%% Technologies: C, C++, Sybase, Excel VBA, ClearCase
\end{Detail}

%% \iffalse   %%% Hide this old job ??
\BigGap
\Entry
\href{http://www.blackrock.com/}
{\textbf{BlackRock Financial Management}}
\hfill 
%% \DatestampYM{1993}{03} --  \DatestampYM{1995}{06}
\DatestampY{1993} --  \DatestampY{1995}

%%\Gap
%%\BulletItem
Software Developer
\hfill
New York, New York

\begin{Detail}
\Item
Developed systems in C/C++ for MBS mortgage, Treasury, and Repo trading desks.
%%\Gap
%%Technologies: C, C++, Sybase, tcl, ksh, Motif, X11
\end{Detail}
%% \fi

\iffalse
\BigGap
\Entry
\textbf{Prophet 21, Inc.}
\hfill 
Yardley, Pennsylvania

%%\Gap
%%\BulletItem
Associate Developer
\hfill
\DatestampYM{1992}{06} --  \DatestampYM{1993}{03}

\begin{Detail}
\BulletItem
Wrote and maintained features for point-of-sale systems targeted to retailers and wholesalers

\Gap
Technologies: Sybase, ksh
\end{Detail}
\fi


%%%%%%%%%%%%%%%
%% EDUCATION %%
%%%%%%%%%%%%%%%

\Section
{Education}
{Education}
{PDF:Education}

\Entry
\href{https://gatech.edu}
{\textbf{Georgia Institute of Technology}}
\hfill Atlanta, Georgia

M.S. in Computer Science
\hfill 
\textit{Expected December 2022}

\BulletItem
Concentrations in Computing Systems and Machine Learning

\Gap

\Entry
\href{http://www.bucknell.edu/}
{\textbf{Bucknell University}}
\hfill Lewisburg, Pennsylvania

%%\Gap
%%\BulletItem
B.S. in Computer Engineering
\iffalse
\hfill
\DatestampYM{1988}{09} --
\DatestampYM{1992}{06}
\fi
\begin{Detail}
\BulletItem
Senior design project creating reference implementation of DHCP protocol for RFCs 1531 and 1534
%%\BulletItem
%%Summer research internship in digital image processing
\BulletItem
Concentrations in Digital Electronics and Geology

%% Technologies: Computer Science, C, VAX assembly, DHCP, TCP-IP, Digital Electronics
\end{Detail}

%%%%%%%%%%%%
%% CERTIFICATIONS %%
%%%%%%%%%%%%

\Section
{Certifications}
{Certifications}
{PDF:Certifications}

\Entry
\href{https://www.appdynamics.com/certifications/}
{\textbf{AppDynamics Certified Implementations Professional}}
\hfill
\DatestampYM{2016}{11}
\begin{Detail}
\begin{quote}
``An individual who attains the AppDynamics Certified Implementation Professional certification has demonstrated that he or she possesses the minimum acceptable level of knowledge and skills required to deploy AppDynamics Controllers, Agents, EUM Servers and Analytics Servers. 
This individual also has experience using AppDynamics APIs to extend and customize the AppDynamics platform.''
\end{quote}
\end{Detail}

%%%%%%%%%%%%%%%%%%%%%%%%%
%% Lecture EXPERIENCE  %%
%%%%%%%%%%%%%%%%%%%%%%%%%


\Section
{Presentations}%% and Mentoring Experience}
{Presentation}%%  and Mentoring Experience}
{PDF:Presentations}%%AndMentoringExperience}

%%%%%%%%%%%%%%%%%%%%%%%%%
\iffalse      %% INCLUDE only for cv style
%%%%%%%%%%%%%%%%%%%%%%%%%
\Entry
\textbf{Adjunct Lecturer}
\hfill
Fall 2017

Rochester Institute of Technology
\begin{Detail}
\Item
Lecturer for a class of 30 undergraduate and graduate students for the course ISTE-120, an introduction to Java course for Information Science and Technology majors.
\end{Detail}

\BigGap
\Entry
\textbf{Summer Co-Op Mentor}
\hfill
Summer 2015

Paychex, Inc.
\begin{Detail}
\Item
After interviewing a number of students, our team brought on a college student for three months. During that time I mentored them on their development tasks, our software development life cycle, version control, and working on a real-world development team. During this time I had the opportunity to give feedback on code, show how things learned in the classroom apply to corporate development, and work through actual software engineering problems.
\end{Detail}
\fi      %% INCLUDE only for cv style

\BigGap
\Entry
\textbf{Axeda Reporting -- A Guide to Service Intelligence} (Trade Conference)
\hfill
\DatestampYMD{2013}{05}{08}

Axeda Connection Conference, Axeda, Inc.
\begin{Detail}
\Item
Live presentation for Axeda customers to review and demonstrate new features added to the Cognos big data reporting product.
\iffalse
\Item
From the conference program:
\begin{quote}
Service Intelligence enables your organization to apply analytics and decision-making, anytime, anywhere. See an overview of Axeda Reporting and what changes are coming in the near future. This session will include live demos of standard reports, as well creating custom reports with Report Studio.
\end{quote}
\fi
\end{Detail}

\iffalse      %% INCLUDE only for cv style
\BigGap
\Entry
\textbf{Java Technical Lead and Team Lead}
\hfill
\DatestampY{2005} -- 
\DatestampY{2010}

Paychex, Inc.
\begin{Detail}
\BulletItem
As one of the technical leads for Java, developers on the extended team would come to me when they had deep technical questions about their code or odd errors that they couldn't track down. I would also have conversations with experienced developers to determine the ``best'' solution for a problem that would meet given technical or scheduling constraints.
\BulletItem
The department I was in created training exercises for new developers to introduce them to the tools and development environment. I worked with each new developer to review their answers to the Java exercises,  answer questions as they worked through them, and working through topics new to them.
\BulletItem
I gave quarterly presentations to the extended team to introduce Java technologies and internal systems. Topics included web application redesigns and base functionality, Java JSP and servlet development, and web testing techniques.
\BulletItem
As a Java team lead, I was responsible for a small team of developers. Since this was the first job after college for some, as we reviewed their code I would see opportunities to mentor them on advanced Java and JavaScript concepts and topics.
\end{Detail}
\fi      %% INCLUDE only for cv style
%%%%%%%%%%%%
%% AWARDS %%
%%%%%%%%%%%%

\Section
{Awards}
{Awards}
{PDF:Awards}
\textbf{Paychex Paragon Award for Resiliency}, Paychex
\hfill
\DatestampYM{2019}{01}
\begin{Detail}
Every year, five projects are chosen across Paychex that demonstrate resiliency, exceptional delivery, operational efficiency, etc.
Our project, "Adaptive Load Balancing", was chosen. It routes requests in F5 BigIP GTM and LTMs across servers and data centers based on server load, real-time statistics, and other measures. Adaptive Load Balancing has reduced outages and improved response times for services using it.
\end{Detail}
\BigGap

\textbf{AppDynamics Scale Award}, AppDynamics
\hfill
\DatestampYM{2015}{11}
\begin{Detail}
Awarded by AppDynamics to the company that was able to quickly deploy the AppDynamics tool across the most complex environment. Our team deployed the AppDynamics tool to over 2,000 JBoss, WebLogic, and Windows IIS servers across production and non-production environments.
\end{Detail}

%%%%%%%%%%%%%%%%%%
%% PUBLICATIONS %%
%%%%%%%%%%%%%%%%%%

%%%%%%%
%%% KEEP THIS SECTION???
%%%%%%%
\iffalse
\Section
{Publications}
{Publications}
{PDF:Publications}

\SubSection
{Technical Editor}
{Technical Editor}
{PDF:TechnicalEditor}

%% MLA format
\Entry
\href{https://www.amazon.com/Oracle-Certified-Associate-ProgrammerStudy/dp/1849687323}
{Reece, Richard M. \textit{Oracle Certified Associate, Java SE 7 Programmer Study Guide}, Packt Publishing, 2012}

\Gap
\Entry
\href
{https://www.amazon.com/Java-Concurrency-Cookbook-FernandezJavier/dp/1849687889}
{Gonzalez, Javier Fernandez. \textit{Java 7 Concurrency Cookbook}, Packt Publishing, 2012}
\fi

%%%%%%%%%%%%%%%%%%%%%%
%% OTHER ACTIVITIES %%
%%%%%%%%%%%%%%%%%%%%%%
\Section
{Other Activities}
{Other Activities}
{PDF:OtherActivities}

\Entry
\textbf{Leader, Boy Scouts and Cub Scouts}
\hfill
\DatestampY{2007} -- Present

\Entry
\textbf{FAA Certified Private Pilot}
%%\hfill
%%\DatestampY{2000}

%%%%%%%%%%%%%%%%%%%%%%%%%%%%%%%%%%%%%%%%
%% THIS IS A SECTION WITH USAGE NOTES %%
%%%%%%%%%%%%%%%%%%%%%%%%%%%%%%%%%%%%%%%%
%%% COMMENT OUT USAGE EXAMPLES
\iffalse
\newpage
% Declare a new group to limit the scope of \color to this section.
\begingroup
\color{red}

\Section
{This is a\newline
Section\newline
With\newline
Usage Notes}
{This is a Section With Usage Notes (For PDF Bookmark)}
{PDF:ThisIsASectionWithUsageNotes:ForPDFLink}

\SubSection
{This is a SubSection}
{This is a SubSection (For PDF Bookmark)}
{PDF:ThisIsASubSection:ForPDFLink}

\Gap
\BulletItem
Use \CodeCommand{Section\{a\}\{b\}\{c\}} and
\CodeCommand{SubSection\{a\}\{b\}\{c\}}
to create sections and subsections, where
\Code{a} is the heading displayed on the page,
\Code{b} is the PDF bookmark heading, and
\Code{c} is the internal PDF link (must be unique).
Sections and subsections will appear in the PDF bookmarks.
Note the CamelCase command names.

\Gap
\BulletItem
Use
\CodeCommand{Entry},
\CodeCommand{BulletItem},
\CodeCommand{SubBulletItem},
\CodeCommand{Item},
\CodeCommand{SubItem},
\CodeCommand{NumberedItem},
etc.,
to create entries in the main body of the CV.

\Gap
\BulletItem
Enclose entry details between
\CodeCommand{begin\{Detail\}} and
\CodeCommand{end\{Detail\}}
so that they are typeset in a smaller font.
\begin{Detail}
\Item
This is an example of entry detail text enclosed in a \Code{Detail} environment.
\end{Detail}

\Gap
\BulletItem
Use \CodeCommand{Gap} and \CodeCommand{BigGap} to insert vertical spaces between entries to improve layout.

\BigGap
\SubSection
{This is Another SubSection}
{This is Another Subsection (For PDF Bookmark)}
{PDF:ThisIsAnotherSubSection:ForPDFLink}

\Gap
\Entry
This is a plain \CodeCommand{Entry},
followed by an \CodeCommand{hfill} and a date range
\hfill
\DatestampYM{2015}{10} --
\DatestampYM{2015}{12}

\Gap
\BulletItem
This is a \CodeCommand{BulletItem}.
\Item
This is an \CodeCommand{Item}, which has no bullet.
Note the alignment with the \CodeCommand{BulletItem} above.

\Gap
\SubBulletItem
This is a \CodeCommand{SubBulletItem}.
\SubItem
This is a \CodeCommand{SubItem}, which has no bullet.
Note the alignment with the \CodeCommand{SubBulletItem} above.

\Gap
\NumberedItem{[42]}
This is a \CodeCommand{NumberedItem}.
Change the value of the macro \CodeCommand{MaxNumberedItem} to adjust the indentation width.

\BigGap
\SubSection
{Line, Paragraph, and Page Breaks}
{Line, Paragraph, and Page Breaks (For PDF Bookmark)}
{PDF:LineParagraphAndPageBreaks:ForPDFLink}

\Gap
\BulletItem
To create a new line within the same paragraph (i.e., preserving the same paragraph indentation), use \CodeCommand{newline} instead of \CodeCommand{\textbackslash};
the latter will reset the paragraph indentation.

\Gap
\BulletItem
To create a new paragraph, use \CodeCommand{par} or simply leave an empty line.
Paragraph indentations (from
\CodeCommand{Entry},
\CodeCommand{BulletItem},
\CodeCommand{SubBulletItem},
\CodeCommand{Item},
\CodeCommand{SubItem},
\CodeCommand{NumberedItem},
etc.) do not carry across different paragraphs.

\Gap
\BulletItem
To create a new page, use \CodeCommand{newpage}.

\BigGap
\SubSection
{Dates}
{Dates (For PDF Bookmark)}
{PDF:Dates:ForPDFLink}

\Gap
\BulletItem
Use the following macros to specify and display dates consistently:
\SubBulletItem
\CodeCommand{DatestampYMD\{yyyy\}\{MM\}\{dd\}}
(e.g., \CodeCommand{DatestampYMD\{2008\}\{01\}\{15\}})
\SubBulletItem
\CodeCommand{DatestampYM\{yyyy\}\{MM\}}
(e.g., \CodeCommand{DatestampYM\{2008\}\{01\}})
\SubBulletItem
\CodeCommand{DatestampY\{yyyy\}}
(e.g., \CodeCommand{DatestampY\{2008\}})

\Gap
\BulletItem
Change the date format option passed to the document class to adjust how dates are displayed throughout the document:
\SubBulletItem
\Code{MMMyyyy} (``Jan~2008'')
\SubBulletItem
\Code{ddMMMyyyy} (``15~Jan~2008'')
\SubBulletItem
\Code{MMMMyyyy} (``January~2008'')
\SubBulletItem
\Code{ddMMMMyyyy} (``15~January~2008'')
\SubBulletItem
\Code{yyyyMMdd} (``2008-01-15'')
\SubBulletItem
\Code{yyyyMM} (``2008-01'')
\SubBulletItem
\Code{yyyy} (``2008'')

\endgroup

%%% COMMENT OUT USAGE EXAMPLES
\fi

\end{Body}

%%%%%%%%%%%
% CV NOTE %
%%%%%%%%%%%

\UseNoteFont%
\null\hfill%
[\textit{\CVNote}]
\end{document}

